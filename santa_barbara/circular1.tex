\magnification=\magstep1
\baselineskip 12pt
\parskip 8pt plus 1.5pt
\parindent=0pt
\bigskipamount = 0.85cm
\medskipamount = 0.42cm
\raggedbottom
\tolerance = 10000
%\vsize=10truein
%\font\big=cmbx10 scaled \magstep4 
%\font\smallsl=cmsl8 at 8pt
\def\Mo{${\rm M_\odot}$}
\def\1item{\hangindent=1.0truecm \hangafter=1 \noindent}
\def\2item{\hangindent=2.0truecm \hangafter=1 \noindent}
\def\3item{\hangindent=3.0truecm \hangafter=1 \noindent}

\centerline{\bf Cluster comparison run}

{\bf A. Model parameters }

{\it Cosmology}:\3item \quad \ $\Omega=1$; $H_0=50$ km s$^{-1}$ Mpc$^{-1}$;
$\sigma_8=0.65$; CDM-like spectrum with $\Gamma=0.25$; \ $\Omega_b=0.1$ 
($\Gamma=\Omega h$ and the transfer function is as given in equation G3 of
BBKS 1996, ApJ, 304, 15).
                       
{\it Boxsize}:\quad\qquad\          $L=64$ Mpc 


{\bf B. Initial conditions} 

{\it Method:}  \3item Constrained Gaussian random field (using the
                      algorithm of Hoffman \& Ribak 1991, Apj (Lett), 
                      380, L5). 

{\it Parameters:} \3item 3-$\sigma$ fluctuation at the centre of the box for a 
      Gaussian filter of radius $r_0 = 10$ Mpc [in $\exp(-0.5(r/r_0)^2)$].

{\it Format:} \3item  Two fields are provided, either of which can be used to
                 set up the initial conditions. We suggest starting at $z=20$.

\qquad \2item    (i)  The dimensionless linear $\delta \rho/\rho$ field tabulated on 
                      a $256^3$ mesh, and normalized to the present.
                      (Thus to get the values of the field at $z=20$ 
                        you just divide by 21.)

\qquad \2item    (ii) Linear theory displacements for $256^3$ points on a
                      cubic mesh. The displacements (extrapolated to the 
                      present) are given in units of the boxsize. (Thus to 
                      get the values at $z=20$, in units of the boxsize at
                      that redshift, you just divide by 21.)

{\bf C. Simulation diagnostics} 

\qquad\qquad\qquad                Output times:     z=8, 4, 2, 1, 0.5, 0                 

Note: X-ray emissivities (per unit volume) should be calculated as 
       ${\cal L}_X=\rho^2 T^{1/2}$, with $\rho$ in \Mo Mpc$^{-3}$ and $T$ in K. 

1. {\it Time evolution:} 

For all output times, extract the central (32 Mpc)$^3$ (comoving) volume,
and make images (ie 2-D arrays of pixel values) for projections along the 
$z$-axis and for the following quantities (each image should have 1024$^2$ 
pixels): 

1.1 Projected dark matter density (in \Mo Mpc$^{-2}$) 

1.2 Projected gas density (in \Mo Mpc$^{-2}$)

1.3 X-ray surface brightness $(\int {\cal L}_X dl$; ${\cal L}_X$ unit as
above; $l$ in Mpc)

1.4 Emission-weighted temperature $(\int {\cal L}_X T dl/\int {\cal L}_X
dl$; T in K; other units as in 1.3)

For each image, give two versions smoothed as follows: 

\qquad \2item     (i)  At a fixed resolution of $r_0=250$ kpc (comoving) 
                  for the Gaussian smoothing kernel, $\exp(-0.5(r/r_0)^2)$ 

\qquad \2item 	  (ii) At the resolution limit of the simulation, as
                  determined by the simulator. 

	The maps in (i) will be used for a uniform comparison 
        of all the models. The maps in (ii) are intended 
        to display the results of each technique in the best possible light.

2. {\it Present day structure} 

At $z=0$ only: 

2.1 Bulk properties:
 
\qquad \1item	For the sphere centred on the cluster of radius, $r_{200}$, 
           such that the mean interior overdensity is 200, obtain: 

\qquad        a) The value of $r_{200}$

\qquad        b) $M_{\rm dm}:$ \ total dark matter mass (in \Mo) 

\qquad        c) $V_{\rm dm}:$ \ {\it rms} velocity of dark matter
                 particles (in km s$^{-1}$)

\qquad        d) $M_{\rm gas}:$ \ total gas mass  (in \Mo )

\qquad        e) $\overline T:$  \quad \  mean (mass weighted) temperature 
                  (in K)

\qquad        f) $U:$ \quad \    total bulk kinetic energy of the gas (in ergs) 

\qquad        g) $L_{\rm tot}= \int_0^{r_{200}} {\cal L}_x dV$ (${\cal
                  L}_X$ units as above; $V$ in Mpc$^{3}$)

\qquad        h) Inertia tensor: ${\cal I} = \sum_i m_i {\bf x}_i {\bf
                 x}_i /\sum_i m_i$ for dark matter and gas (in Mpc$^2$) 

2.2 Differential properties:
 
\qquad        In 15 spherical shells of logarithmic width 0.2 dex and inner
              radii $10 {\rm kpc} \le r \le 10 {\rm Mpc}$, obtain:

\qquad	a) $\rho_{\rm dm}(r):$    dark matter density profile (in \Mo
                                  Mpc$^{-3}$) 

\qquad	b) $\sigma_{\rm dm}(r):$  dark matter velocity dispersion profile
                                  (in km s$^{-1}$) 

\qquad	a) $\rho_{\rm gas}(r):$   gas density profile (in \Mo Mpc$^{-3}$)

\qquad	c) $T(r):$ \quad           mass-weighted gas temperature profile (in K)

\qquad	d) ${\cal L}_x(r):$        ``X-ray  luminosity" profile
                       calculated as the total luminosity in each bin, 
                       divided by its volume (in units as above).
\vfill\eject
2.3 Other properties:

\qquad \1item  Obtain 1024$^2$ pixel maps of the X-ray surface brightness, 
	   projected down the three principal axes, for a cube of side 
           10~Mpc centred on the cluster. This map should be at the
           resolution at which each technique is deemed reliable by its 
           authors. (Please let us know what this effective resolution is.) 

\vfill\eject\bye 


