\magnification=\magstep1
\baselineskip 12pt
\parskip 8pt plus 1.5pt
\parindent=0pt
\bigskipamount = 0.85cm
\medskipamount = 0.42cm
\raggedbottom
\tolerance = 10000
%\vsize=10truein
%\font\big=cmbx10 scaled \magstep4 
%\font\smallsl=cmsl8 at 8pt
\def\Mo{${\rm M_\odot}$}
\def\1item{\hangindent=1.0truecm \hangafter=1 \noindent}
\def\2item{\hangindent=2.0truecm \hangafter=1 \noindent}
\def\3item{\hangindent=3.0truecm \hangafter=1 \noindent}

Dear colleague,

You will recall that earlier this Spring, at the Santa Barbara
workshop, we agreed to carry out a simulation of the formation of a galaxy
cluster using our various codes. The aim of this exercise is
twofold. Firstly, we want to see which (if any) properties are reproducible
with different codes. Secondly, we want to create a ``fiducial" simulation
to be used as a standard benchmark test for future codes. As a first step,
we agreed to do the simplest case: the growth of a cluster containing only 
dark matter and a non-radiative gas, evolved from realistic initial
conditions.

We offered to generate a set of initial conditions designed to be flexible
enough so that they could be easily adapted to suit the different codes. We
have now done so (with help from Shaun Cole) and we have simulated a $64^3$
dark matter only version; the cluster ends up containing about 6\% of the
total mass in the simulation and does not undergo any major mergers after
$z\sim 0.5$. A postcript file with plots of the dark matter distibution
will soon be available through the Net.

We also offered to coordinate the analysis and comparison of the
results. For this to work effectively, we propose below a number of simple
diagnostics (which should be calculated by each simulator) but which are
detailed enough to provide a suitable comparison of the codes. We will
simply collate and plot the results. Note that we want to focus on the
properties of the cluster itself and on its history. Many techniques may
give poor resolution on regions well outside the cluster, but this is not
to be considered a serious disadvantage for the present comparison. We
anticipate that most people will provide a typical ``production run''
simulation for this project. 

We are circulating this, in the first instance, to those of you who have
given input to the project and who have already ``agreed'' to participate
in it: 

Dick Bond and James Wadsley \hfill\break 
Renyue Cen \hfill\break
Gus Evrard \hfill\break
Neal Katz, George Lake and Lars Hernquist \hfill\break
Julio Navarro \hfill\break
Mike Norman \hfill\break
Jerry Ostriker, U Pen, Nick Gnedin \hfill\break
Hugh Couchman, Peter Thomas and Frazer Pearce \hfill\break
Matthias Steinmetz \hfill\break

We propose the following timetable:

{\bf 1) Immediately :}  \quad Let us know if you still intend to take 
part.

{\bf 2) June 30:} \3item \ Send comments on this proposal. Ideally, the 
                  initial conditions are acceptable since they were arrived at
                  after discussions with most of the participants. 
                   Comments are welcome, however, on the proposed set of
                   diagnostics and on the overall organization of the project.
 
{\bf 3) July 15:}  \3item \  We will send an agreed revised version of the 
                    proposal and make all required files available over the
                    Net through the Durham Home Page. 

{\bf 4) December 31 :} \ \ \  Completion of runs.

{\bf 5) January 96  :} \quad \ \ Analysis and comparison.

{\bf 6) February 96 :}    \quad Preparation of a paper.

\bigskip

Carlos Frenk \hfill\break
Simon White 
\vfill\eject\bye 


